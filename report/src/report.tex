\documentclass[a4paper]{article}

\usepackage[a4paper, top=2cm, bottom=2cm, left=2cm, right=2cm]{geometry}
\usepackage[T1, T2A]{fontenc}
\usepackage[fontsize=14pt]{fontsize}

\usepackage[english, russian]{babel}
\usepackage[utf8]{inputenc}

\usepackage{hyperref}
\usepackage{indentfirst}

\hypersetup{pdfpagemode=FullScreen, colorlinks=true, linkcolor=black, urlcolor=cyan}

\begin{document}
	\thispagestyle{empty}
	
	\begin{center}
	{\LARGE \textsc{НОВОСИБИРСКИЙ ГОСУДАРСТВЕННЫЙ УНИВЕРСИТЕТ}\par}
	{\textsc{ФАКУЛЬТЕТ ИНФОРМАЦИОННЫХ ТЕХНОЛОГИЙ}\par}
	
	\vspace{3cm}
	
	{\huge\bfseries Теоретические основы обработки информации\par}


	
	\vspace{1cm}
	
	{\Large\bfseries Алгоритм Pagerank\par}
	
	\vspace{10cm}
	
	\begin{flushright}
		Кондренко К.П, Неретин С.И., группа 21203
	\end{flushright}
	
	\vfill
	
	{\large \today\par}
\end{center}

		
	\newpage
		
	\tableofcontents
	
	\pagestyle{plain}
	
	\newpage
	
	\section{Введение}
	
	Ранжирование веб-страниц является одним из ключевых аспектов поисковых систем, так как позволяет предоставлять пользователям наиболее релевантные результаты в ответ на их запросы. Алгоритм PageRank является одним из наиболее известных и широко используемых методов ранжирования веб-страниц на основе их связности ссылками. Однако существуют и другие подходы к решению этой задачи, которые могут быть эффективны в различных контекстах. В данном обзоре рассматриваются другие методы ранжирования веб-страниц, их особенности, преимущества и недостатки.
	
\section{Описание алгоритма PageRank}

Алгоритм PageRank является одним из основных алгоритмов ранжирования веб-страниц, который используется в поисковых системах для определения их важности и релевантности для конкретного запроса пользователя. Он был разработан Ларри Пейджем и Сергеем Брином при создании поисковой системы Google.

В основе алгоритма лежит представление веба в виде графа, где узлами являются веб-страницы, а рёбра представляют собой гиперссылки между этими страницами. PageRank рассматривает каждую страницу как узел в этом графе и присваивает ей числовую оценку, которая отражает ее важность.

Принцип работы алгоритма можно описать следующим образом:

\begin{enumerate}
	\item \textbf{Инициализация рангов}: В начале каждая страница инициализируется с одинаковым начальным рангом. Это может быть равномерное распределение или другой метод.
	\item \textbf{Распространение ранга}: Затем выполняется итерационный процесс, в ходе которого ранги страниц обновляются на основе их собственного ранга и рангов страниц, которые на них ссылаются. Более важные страницы (те, которые имеют больше входящих ссылок) передают больше своего ранга страницам, на которые они ссылкаются.
	\item \textbf{Учет демпинг фактора}: Для предотвращения переполнения и улучшения качества ранжирования вводится демпинг фактор (обычно около 0.85). Он определяет вероятность того, что пользователь перейдет на другую страницу, вместо того чтобы продолжать переходить по ссылкам.
	\item \textbf{Конвергенция}: Процесс итераций продолжается до тех пор, пока ранги страниц не стабилизируются, то есть до тех пор, пока изменения рангов страниц между итерациями не станут незначительными.
\end{enumerate}

Важно отметить, что PageRank не является единственным фактором ранжирования, используемым в поисковых системах, но он является одним из ключевых и может влиять на позицию веб-страниц в результатах поиска. Алгоритм был дополнен и улучшен с течением времени, чтобы более точно отражать реальные потребности пользователей и бороться с различными видами спама и манипуляций.

	
\section{Специфика задачи ранжирования}

Ранжирование веб-страниц является критически важной задачей для поисковых систем, однако аналогичные принципы ранжирования могут быть применены и в других областях, например, в обработке текстов. Один из примеров такого применения - алгоритм TextRank, который используется для автоматического создания кратких содержаний текстов.

\section{О Алгоритме TextRank и как он основан на Pagerank}

Textrank основан на предположении, что важные предложения в тексте часто содержат ключевую информацию и, следовательно, могут рассматриваться как важные "страницы" в графе предложений. Принцип работы Textrank аналогичен алгоритму PageRank, где вместо веб-страниц рассматриваются предложения текста, а вместо ссылок - связи между предложениями. Таким образом, TextRank позволяет определить наиболее важные предложения в тексте на основе их взаимосвязей.



	\section{Анализ существующих подходов}
	
	\subsection{HITS}
	
Алгоритм HITS (Hyperlink-Induced Topic Search) также используется для ранжирования веб-страниц, но в отличие от PageRank, HITS оценивает не только важность страницы как источника информации (авторитетности), но и как источника ссылок (хабовости). Кратко описывая алгоритм HITS:

\begin{enumerate}
	\item Инициализация всех страниц как хабовых и авторитетных.
	\item Итеративное обновление хабовости и авторитетности страниц.
	\item Рекурсивное обновление рангов до сходимости.
	\item Получение конечных значений хабовости и авторитетности для каждой страницы.
\end{enumerate}

Разница между алгоритмами заключается в том, что PageRank оценивает важность страницы на основе ее собственных характеристик и характеристик страниц, которые на нее ссылаются, в то время как HITS учитывает как важность страницы, так и ее активность как источника ссылок.

	\subsection{BM25}
	
BM25 (Best Matching 25) представляет собой вероятностный метод ранжирования, используемый в информационном поиске для оценки релевантности документов по отношению к запросу пользователя. Этот алгоритм является усовершенствованным вариантом модели TF-IDF (Term Frequency-Inverse Document Frequency), который учитывает не только частоту терминов в документе, но и другие факторы, такие как длина документа и отдельные компоненты запроса.

Важным преимуществом BM25 является его способность эффективно обрабатывать как короткие, так и длинные текстовые документы и запросы. В отличие от алгоритмов ранжирования веб-страниц, таких как HITS и PageRank, которые оценивают важность веб-страниц на основе их авторитетности и структуры ссылок между ними, BM25 оценивает релевантность документов на основе совпадения между терминами запроса и содержанием документа, учитывая различные факторы для достижения наилучшего соответствия запросу.

	\subsection{Системы на основе машинного обучения}
	
	С развитием методов машинного обучения появились и методы ранжирования, основанные на обучении моделей на больших наборах данных. Например, вместо использования только структуры графа ссылок, данные о поведении пользователей и текстовое содержимое страниц могут быть использованы для создания модели ранжирования. Методы машинного обучения также могут учитывать контекст запроса пользователя и другие факторы, что делает их более адаптивными к изменяющимся требованиям пользователей.
	\newpage
	\section{Актуальность и специфика подходов}
	
	Каждый из перечисленных подходов имеет свои преимущества и недостатки, а также свою область применения. Например, алгоритм PageRank хорошо работает на больших графах сильно связанных страниц, но может быть менее эффективен на страницах с малым количеством ссылок. Алгоритм HITS может быть более подходящим для поиска конкретных тематических сообществ в веб-графе.
	
	Системы на основе машинного обучения обычно требуют большого количества данных для обучения, но могут быть более гибкими и точными в ранжировании. Они также могут лучше адаптироваться к изменяющимся требованиям пользователей и окружению. Использование данных из социальных сетей может повысить релевантность ранжирования, учитывая социальные факторы и поведение пользователей.
	\\
	\\
	\section{Заключение}
	
	В заключение, ранжирование веб-страниц является сложной задачей, и существует множество подходов к ее решению. Каждый из этих подходов имеет свои особенности, преимущества и недостатки. Выбор подхода зависит от конкретных требований и условий задачи. Дальнейшие исследования в этой области могут привести к созданию более эффективных методов ранжирования и улучшению качества поисковых систем.

\newpage
\section{Литература}	
\begin{itemize}
	\item Brin, S., Page, L. (1998). The Anatomy of a Large-Scale Hypertextual Web Search Engine.
	\item Kleinberg, J. M. (1999). Authoritative sources in a hyperlinked environment.
	\item Langville, A. N., \& Meyer, C. D. (2006). Google's PageRank and beyond: The science of search engine rankings.
	\item Liu, B. (2011). Web Data Mining: Exploring Hyperlinks, Contents, and Usage Data.
	\item Manning, C. D., Raghavan, P., \& Schütze, H. (2008). Introduction to information retrieval.
	\item Richardson, M., \& Domingos, P. (2002). The intelligent surfer: Probabilistic combination of link and content information in PageRank.
	\item Wu, S., \& Wen, J. R. (2011). Learning to rank for information retrieval.
	\item https://habr.com/ru/articles/533096/
\end{itemize}
	
	
\end{document}
